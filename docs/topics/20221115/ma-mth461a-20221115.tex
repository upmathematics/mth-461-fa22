% Options for packages loaded elsewhere
\PassOptionsToPackage{unicode}{hyperref}
\PassOptionsToPackage{hyphens}{url}
\PassOptionsToPackage{dvipsnames,svgnames,x11names}{xcolor}
%
\documentclass[
]{article}
\usepackage{amsmath,amssymb}
\usepackage{lmodern}
\usepackage{iftex}
\ifPDFTeX
  \usepackage[T1]{fontenc}
  \usepackage[utf8]{inputenc}
  \usepackage{textcomp} % provide euro and other symbols
\else % if luatex or xetex
  \usepackage{unicode-math}
  \defaultfontfeatures{Scale=MatchLowercase}
  \defaultfontfeatures[\rmfamily]{Ligatures=TeX,Scale=1}
\fi
% Use upquote if available, for straight quotes in verbatim environments
\IfFileExists{upquote.sty}{\usepackage{upquote}}{}
\IfFileExists{microtype.sty}{% use microtype if available
  \usepackage[]{microtype}
  \UseMicrotypeSet[protrusion]{basicmath} % disable protrusion for tt fonts
}{}
\makeatletter
\@ifundefined{KOMAClassName}{% if non-KOMA class
  \IfFileExists{parskip.sty}{%
    \usepackage{parskip}
  }{% else
    \setlength{\parindent}{0pt}
    \setlength{\parskip}{6pt plus 2pt minus 1pt}}
}{% if KOMA class
  \KOMAoptions{parskip=half}}
\makeatother
\usepackage{xcolor}
\usepackage[margin=1in]{geometry}
\usepackage{longtable,booktabs,array}
\usepackage{calc} % for calculating minipage widths
% Correct order of tables after \paragraph or \subparagraph
\usepackage{etoolbox}
\makeatletter
\patchcmd\longtable{\par}{\if@noskipsec\mbox{}\fi\par}{}{}
\makeatother
% Allow footnotes in longtable head/foot
\IfFileExists{footnotehyper.sty}{\usepackage{footnotehyper}}{\usepackage{footnote}}
\makesavenoteenv{longtable}
\usepackage{graphicx}
\makeatletter
\def\maxwidth{\ifdim\Gin@nat@width>\linewidth\linewidth\else\Gin@nat@width\fi}
\def\maxheight{\ifdim\Gin@nat@height>\textheight\textheight\else\Gin@nat@height\fi}
\makeatother
% Scale images if necessary, so that they will not overflow the page
% margins by default, and it is still possible to overwrite the defaults
% using explicit options in \includegraphics[width, height, ...]{}
\setkeys{Gin}{width=\maxwidth,height=\maxheight,keepaspectratio}
% Set default figure placement to htbp
\makeatletter
\def\fps@figure{htbp}
\makeatother
\setlength{\emergencystretch}{3em} % prevent overfull lines
\providecommand{\tightlist}{%
  \setlength{\itemsep}{0pt}\setlength{\parskip}{0pt}}
\setcounter{secnumdepth}{-\maxdimen} % remove section numbering
\usepackage{fancyhdr}
\pagestyle{fancy}
\fancyhead[RO,R]{MTH 461A - Fall 2022}
\fancyfoot[CO,C]{}
\fancyfoot[R]{\thepage}
\usepackage{float}

\ifLuaTeX
  \usepackage{selnolig}  % disable illegal ligatures
\fi
\IfFileExists{bookmark.sty}{\usepackage{bookmark}}{\usepackage{hyperref}}
\IfFileExists{xurl.sty}{\usepackage{xurl}}{} % add URL line breaks if available
\urlstyle{same} % disable monospaced font for URLs
\hypersetup{
  pdftitle={Moment Generating Functions for CRVs},
  colorlinks=true,
  linkcolor={Maroon},
  filecolor={Maroon},
  citecolor={Blue},
  urlcolor={blue},
  pdfcreator={LaTeX via pandoc}}

\title{\textbf{Moment Generating Functions for CRVs}}
\usepackage{etoolbox}
\makeatletter
\providecommand{\subtitle}[1]{% add subtitle to \maketitle
  \apptocmd{\@title}{\par {\large #1 \par}}{}{}
}
\makeatother
\subtitle{Mini-Assignment 2022-11-10 - MTH 461 - Fall 2022}
\author{}
\date{\vspace{-2.5em}}

\begin{document}
\maketitle

\hfill\break

\textbf{Instructions:}

\begin{itemize}
\item
  Please provide complete solutions for each problem. If it involves mathematical computations, explanations, or analysis, please provide your reasoning or detailed solutions.
\item
  Note that some problems have multiple solutions or ways to solve it. Make sure that your solutions are clear enough to showcase your work and understanding of the material.
\item
  Creativity and collaborations are encouraged. Use all of the resources you have and what you need to complete the mini-assignment. Each student must take personal responsibility and submit their work individually. Please abide by the University of Portland Academic Honor Principle.
\item
  There are two ways you can write your answers, a: by handwriting (either physically or digitally), or b: by typing on a template document with file type options, which can be downloaded from the course website.
\item
  If you had handwritten your answers/solutions on a physical paper, make sure to label it properly and please scan your document using a scanner app for convenience. Suggestions: (1) \href{https://play.google.com/store/apps/details?id=com.appxy.tinyscanner\&hl=en_US\&gl=US}{``Tiny Scanner'' for Android} or (2) \href{https://apps.apple.com/us/app/scanner-app-scan-pdf-document/id595563753}{``Scanner App'' for iOS}.
\item
  \textbf{Please save your work as one pdf file, don't put your name in any part of the document, and submit it to the Teams Assignments for this course. Your document upload will correspond to your name automatically in Teams.}
\item
  If you have questions or concerns, please feel free to ask the instructor.
\end{itemize}

\newpage

\begin{enumerate}
\def\labelenumi{\arabic{enumi}.}
\item
  Let \(X\) and \(Y\) be two independent random variables with respective moment generating functions
  \[M_X(s) = \frac{1}{1-5s}, \text{ if } s < \frac{1}{5}, \text{ and }\]
  \[M_Y(s) = \frac{1}{(1-5s)^2}, \text{ if } s < \frac{1}{5}.\]
  Find \(E[(X+Y)^2]\).
\item
  Suppose \(X\) and \(Y\) are independent Poisson random variables with parameters \(\lambda_x\), \(\lambda_y\), respectively. Find the distribution of \(X + Y\). (Hint: Find the Joint MGF first, and then match the results to known MGFs.)

  Recall that the mgf for a Poisson random variable is given by
  \[M_x(s) = e^{\lambda (e^s - 1)}.\]
\item
  True or False? If \(X \sim Exponential(\lambda_x)\) and \(Y \sim Exponential(\lambda_y)\) then \(X + Y \sim Exponential(\lambda_x + \lambda_y)\). Justify your answer. (Hint: Check first if X and Y are independent, find the joint MGF, then match results to known MGFs.)

  Recall that the mgf for an Exponential random variable is given by
  \[M_X(s) = \frac{\lambda}{\lambda - s}.\]
\end{enumerate}

\end{document}
